\chapter{Conclusions}
\label{Cap6}

Liquid State Theory is at the core of most modern Soft Matter research fields. As such, it 
seems important to clearly understand its underlying theoretical frameworks and their main 
results. Also, it is important to have a clear understanding of the methods used to provide 
accurate solutions to specific problems. Without the exact theoretical framework of the 
Ornstein-Zernike equation, most Colloidal Soft Matter research fields would struggle with 
other theories which are more complicated and harder to implement. Rather, the 
Ornstein-Zernike equation is simple, effective, and has the outstanding ability to be able 
to generalize well, beyond its original goal, which was the description of simple liquids. 
It is the purpose of this thesis to showcase the richness of the Ornstein-Zernike 
formalism, and to inject novel ways of thinking about it, using it, and most importantly, 
of solving it.

Throughout this thesis, the Ornstein-Zernike formalism was introduced as an integral 
equation theory that is capable of describing the structure of a simple liquid, in 
particular we focused on the fluid model that results the cornerstone of the Statistical 
Mechanics of fluids,  namely,  the hard-sphere fluid. Using the theory of equilibrium 
Statistical Mechanics, the Ornstein-Zernike equation was described, along with the 
important quantity, the \emph{bridge function}. This function plays a fundamental role in 
trying to solve the Ornstein-Zernike equation, because it provides a closed expression  for 
the indirect correlations between particles the system that is the subject of study. 
However, the reality is that this function is still an approximation, and to correctly 
describe a particular fluid, this approximation must be chosen arbitrarily, based primarily 
on computer simulations and previous research. But this seems cumbersome, 
and although lots of literature reflect this, it is still used based on experience and 
common knowledge. This is because finding an exact bridge function is very hard, it cannot 
be done either numerically or analytically, thus approximations are needed if a solution is 
wanted for the fluid under scrutiny.

For this reason, Computational Intelligence methods are introduced in this thesis. First, 
instead of trying to approximate the bridge function, a more general approximation is 
presented in the form of a Neural Network. Due to the 
\emph{Universal Approximation Theorem}~\cite{hornikMultilayerFeedforwardNetworks1989,hornikApproximationCapabilitiesMultilayer1991,cybenkoApproximationSuperpositionsSigmoidal1989}, 
Neural Networks can approximate any continuous and well-defined, smooth function. Using 
this knowledge, instead of defining a particular approximation for the bridge function, a 
new way of solving the Ornstein-Zernike equation is devised, that attempts to use the 
Neural Network to find the best approximation for the hard sphere fluid. The results show 
that the Neural Network reduces to a particular bridge function approximation, the 
Hypernetted Chain solution, for the case of the hard sphere fluid. This points toward to 
the direction that without any more information from the system, the Neural Network tries 
to stay in a particular minimum, which in this case corresponds to the Hypernetted Chain 
solution. However, this also means that if there is more information about the underlying 
Physics of the systems, more information can be used along with the Neural Network, 
potentially given a better solution to the problem.

Instead of relying on a universal approximator, one can effectively use the underlying 
Physics of the problem, and use different Computational Intelligence methods that can solve 
the same problem. This proposal is shown in this thesis, using the Verlet bridge function 
approximation, together with the Kinoshita variation, to provide more information to the 
Ornstein-Zernike equation. Nevertheless, instead of using these bridge functions as they 
are, the whole problem was turned into a black-box optimization problem that attempts to 
find the best variation of the Kinoshita bridge function that can also provide partial 
thermodynamic consistency to the problem. It seems that thermodynamic consistency is a 
better way of dealing with the problem, and the results show this. The black-box 
optimization problem was solved using Natural Evolution Strategies, a type of optimization 
method that comes directly from the family of Evolutionary Computation optimization 
algorithms. When computing the isothermal compressibility for different density values of 
the hard sphere fluid, the results obtained from these optimization methods provides 
excellent agreement with the theoretical results. This means that, by reformulating the 
problem to an optimization problem with less parameters to find, the solution is better and 
more accurate. One of the drawbacks of this method is that it is slow and computationally 
expensive. Nonetheless, the method can be extended to other systems and interaction 
potentials.

\section{Future work}
To close this thesis, a few ideas will be presented which can serve as future work that can 
carry on from the main proposals presented in this work. Some of these ideas serve to close 
off the points made in the results presented, and others might serve as inspiration for the 
reader interested in the topics developed in this work.

\subsection{A functional approach to the bridge function}
In \autoref{Cap4}, the bridge function was presented as a series expansion on the density 
in \autoref{eq:expansion-densidad}. A different idea, related to this one, is that the 
approximation can be worked out from a series expansion where the terms are themselves 
neural networks. This would look like,
\begin{equation}
    b \left[r, N_{\theta}(r)\right] = \sum_{i=1}^{\infty} a_{i} \, N_{\theta \, , i} (r)
    \; ,
    \label{eq:functional-bridge}
\end{equation}
where the notation \(N_{\theta,i}(r)\) represents a different neural network for each 
\(i\) in the expansion. In other words, instead of having just a polynomial series 
expansion in the bridge function, the terms themselves would be constructed from 
\emph{functions}, thus providing a \emph{functional approach} to the bridge function. This 
would have all the mathematical power and generalization of simple neural networks, and 
provide more information to the bridge function itself. However, this is not a simple 
proposal and the mathematical framework to work this out might be daunting and difficult to 
use.

\subsection{Surrogate-based modeling}
The Ornstein-Zernike equation can be solved efficiently using the Fast Fourier 
Transform~\cite{hammingNumericalMethodsScientists2012}. But sometimes the solution is 
difficult to obtain using the same iterative Piccard scheme. Further, if the black-box 
optimization method is implemented to enforce partial thermodynamic consistency, then the 
solution can become difficult to obtain. A solution to this problem is to propose a 
\emph{surrogate-based modeling} approach. Surrogate 
models~\cite{forresterRecentAdvancesSurrogatebased2009} as substitute models that 
\emph{cleverly} interpolate between the input and output of difficult-to-evaluate 
functions. The meaning of clever here means that most of these substitute models rely on 
modern Machine Learning techniques, such as Neural Networks, Gaussian Processes, and many 
others.

What would happen if, instead of solving the Ornstein-Zernike equation directly, a 
surrogate model could be built from previous known answers? At least for simple state 
points, the surrogate model could be expected to perform well. But when difficult phenomena 
arise, such as phase transitions, spinodal lines and others, it might be unreasonable to 
expect too much from the surrogates. For this, additional information based on the 
underlying Physics might serve helpful in defining better surrogate models. This can also 
help in the optimization process of finding better parameters for the bridge function 
approximation.

\subsection{Closing remarks}
Finally, it is important to note that the use of Computational Intelligence methods in 
Liquid State Theory is novel, and is still in its exploration phase. Modern Soft Matter 
research has been constantly adopting new methods brought from the Machine Learning and 
Deep Learning communities, and this is creating a trend in this research field. This is 
good, because Machine Learning methods, and in general Computational Intelligence methods, 
have proven useful in cases where one can use all the previous Physics together with more 
powerful and robust methods, such that new ways of dealing with the same problems arise. 
These new ways can be better, more efficient, or simply put, easier to understand and 
implement. It is the main purpose of this thesis to showcase this important idea: that new 
Computational Intelligence methods and technologies can help boost modern Soft Matter and 
Liquid State Research, as long as the elemental Physics of the system are used 
appropriately. Hopefully, in years to come, these ideas will become the standard way of 
dealing with the fantastic research in Liquid State Theory and Soft Matter Physics.