\chapter{Evolutionary optimization for the Kinoshita closure}
\label{Cap5}

In the previous chapter, a neural network was used to approximate the bridge function in 
the solution to the Ornstein-Zernike (OZ) equation. From the results obtained, it was 
implied that, without any more physical information, the neural network reduced its 
approximation to the well-known Hypernetted Chain closure relation. In this chapter, the 
modified Verlet bridge function is introduced, along with its variation, the Kinoshita 
closure. This bridge function contains more information in terms of the indirect 
correlation function, \(\gamma(r)\), which provides a more accurate solution when dealing 
with the hard-sphere fluid. Still, the Kinoshita closure is not thermodynamically 
consistent. Thus, in this chapter, the proposal of parametrizing the Kinoshita is 
introduced. These parameters are then fitted using Evolutionary Computation (EC) method for 
derivative-free optimization tasks. This is in the same spirit as with the 
Rogers-Young~\cite{rogersNewThermodynamicallyConsistent1984b} and the 
Zerah-Hansen~\cite{zerahSelfConsistentIntegral1986} closure relations. The goal of this 
proposal is to alleviate the problems that using a neural network has, in terms of 
incoporating more Physics into the modeling, instead of relying on the neural network to 
learn something by its own.

\section{The Kinoshita closure and its parametrization}
\section{Thermodynamic consistency}
\section{Black-box optimization problem implementation}
\section{Results}
\section{Discussion}
\section{Concluding remarks}