\chapter{Evolutionary optimization for the Kinoshita closure}
\label{Cap5}

In the previous chapter, a neural network was used to approximate the bridge function in 
the solution to the Ornstein-Zernike (OZ) equation. From the results obtained, it was 
implied that, without any more physical information, the neural network reduced its 
approximation to the well-known Hypernetted Chain closure relation. In this chapter, the 
modified Verlet bridge function is introduced, along with its variation, the Kinoshita 
closure. This bridge function contains more information in terms of the indirect 
correlation function, \(\gamma(r)\), which provides a more accurate solution when dealing 
with the hard-sphere fluid. Still, the Kinoshita closure is not thermodynamically 
consistent. Thus, in this chapter, the proposal of parametrizing the Kinoshita is 
introduced. These parameters are then fitted using Evolutionary Computation (EC) method for 
derivative-free optimization tasks. This is in the same spirit as with the 
Rogers-Young~\cite{rogersNewThermodynamicallyConsistent1984b} and the 
Zerah-Hansen~\cite{zerahSelfConsistentIntegral1986} closure relations. The goal of this 
proposal is to alleviate the problems that using a neural network has, in terms of 
incoporating more Physics into the modeling, instead of relying on the neural network to 
learn something by its own.

\section{The Kinoshita closure and its parametrization}
The proposal in this chapter deals with the modified Verlet closure and a variation of 
this, the Kinoshita modification. Both of these closure relations have been formulated to 
deal with the hard-sphere fluid, for low and large density values. The purpose of this 
section is to look at these closure relations more closely, as well as to define the 
parametrization to be used in later computations and results.

\subsection{The modified Verlet bridge function approximation}
Between 1980 and 1981, Loup Verlet published two papers were he introduced what is now 
called the \emph{modified Verlet} bridge 
function~\cite{verletIntegralEquationsClassical1980,verletIntegralEquationsClassical1981}. 
The idea behind his bridge function approximation was to derive a good approximation based 
on reproducing the exact first five virial coefficients of the hard sphere equation of 
state. The original proposition for the bridge function by Verlet was,
\begin{equation}
    B(r) = - \frac{A \, \gamma^{2}(r) \, / 2}{1 + B \, \gamma(r) \, / 2}
    \; .
    \label{eq:verlet-params}
\end{equation}
By fitting the exact values of the virial coefficients, Verlet reached the conclusion that 
the values for \(A, B\) that could reproduce the results for the hard-sphere fluid up to 
the fluid-solid transition where the values of,
\begin{equation}
    A = 1 \, , \quad B = \frac{4}{5}
    \; ,
    \label{eq:ab-verlet}
\end{equation}
which turns the original closure relation in \autoref{eq:verlet-params} into,
\begin{equation}
    B(r) = - \frac{0.5 \, \gamma^{2}(r)}{1 + 0.8 \, \gamma(r)}
    \; .
    \label{eq:mVerlet}
\end{equation}
This turned out to be an accurate bridge function approximation for the case of the 
hard-sphere fluid, even when hard sphere mixtures are 
considered~\cite{lopez-sanchezDemixingTransitionStructure2013a}.
Some of the advantages of this approximation is the fact that it is simple, efficient and 
there are no free parameters to be fixed, as is the case in the 
Rogers-Young~\cite{rogersNewThermodynamicallyConsistent1984b} and the 
Zerah-Hansen~\cite{zerahSelfConsistentIntegral1986} closure relations. The problem lies, as 
with other closure relations, with the fact that the bridge function in 
\autoref{eq:mVerlet} is not thermodynamically consistent. There are other closure relations 
which are already thermodynamically consistent, as is the case of the reference Hypernetted 
Chain approximation~\cite{ladoSolutionsReferencehypernettedchainEquation1983}. This 
approximation comes from first principles, and the thermodynamic consistency comes from the 
fact that the reference potential is chosen such that the free energy of the fluid is 
minimized. Still, the problem with this closure relation, although accurate, is that is 
might be hard to implement and use in several different scenarios.

\subsection{The Kinoshita variation}
For this reason, the idea of this chapter is to use the modified Verlet approximation, and instead of using the fixed parameters found from theoretical arguments, to let an evolutionary algorithm find the best ones that can also establish thermodynamic consistency. For this reason, the Kinoshita variation~\cite{kinoshitaInteractionSurfacesSolvophobicity2003} to the modified Verlet closure relation is introduced. This variation reads,
\begin{equation}
    B(r) = - \frac{0.5 \, \gamma^{2}(r)}{1 + 0.8 \, \left\lvert \gamma(r) \right\rvert}
    \; ,
    \label{eq:kinoshita}
\end{equation}
and introduces the absolute value \(\left\lvert \cdot \right\rvert\) in the denominator, 
which increases the numerical stability of the closure relation for the case when there are 
very large and negative values of \(\gamma(r)\). However, in this case, the idea is to leave the original form of the closure relation, that is with two free paramaters,
\begin{equation}
    B(r) = \frac{\alpha \, \gamma^{2}(r)}{1 + \beta \, \left\lvert \gamma(r) \right\rvert}
    \; ,
    \label{eq:kinoshita-params}
\end{equation}
and instead look for the best values of both \(\alpha, \beta\) that yield an accurate 
result for the case of the isotropic hard sphere fluid in three dimensions.
To find these values, partial thermodynamic consistency will be enforced through the use of 
the pressure equations, which shall be discussed next.

\section{Thermodynamic consistency}
\section{Black-box optimization problem implementation}
\section{Results}
\section{Discussion}
\section{Concluding remarks}