\chapter{Introduction}
\label{Cap1}

Since the mainstream adoption of \emph{Machine Learning} (ML) methods
on common tasks such as object recognition, face identification,
and human-computer interactions~\cite{lecunDeepLearning2015},
scientists have tried to adopt most of these techniques to further research
in their respective fields. From drug development~\cite{redaMachineLearningApplications2020}
to genetics and biotechnology~\cite{libbrechtMachineLearningApplications2015},
multiple applications of ML to current important research fields have seen
widespread interest for their generalization and automatic discovery attributes.
It is with the inspiration from these applications that physicists have attempted to use 
such methods in diverse Physics fields~\cite{carleoMachineLearningPhysical2019a,dunjkoMachineLearningArtificial2018,carrasquillaMachineLearningPhases2017a}.

Most of the attempts and successes of using ML methods in Physical sciences come from
the direct application of common ML pipelines and uses, such as \emph{classification},
\emph{regression}, and \emph{unsupervised learning}~\cite{hastieElementsStatisticalLearning2009}, just to name some.
Such is the case of the determination of the
critical point of the Ising model as means of a classification task~\cite{carrasquillaMachineLearningPhases2017a}.
Similar is the case the use of \emph{computer vision} and \emph{deep learning} techniques in
particle physics have seen great applications when dealing with experimental data~\cite{radovicMachineLearningEnergy2018}.
In each of the previous examples, scientists have taken the most common applications
of ML methods and have adjusted them for their respective research problems.
This has the advantage that such ML techniques have been extensively researched
and developed, so physicists know that these methods are robust and useful for
the problems the have been developed for.
However, it turns out that not all ML techniques can be readily applied to they problem
at hand, and one should instead try to use the Physics of the problem and use it along
with the ML method to boost its usefulness and flexibility.