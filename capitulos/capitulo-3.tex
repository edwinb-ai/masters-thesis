\chapter{Computational Intelligence and Machine Learning}
\label{Cap3}

In this chapter, the fundamentals of Computational Intelligence and Machine Learning 
are developed. Particularly, the focus of the chapter is to present the main tools 
used in this thesis, namely \emph{neural networks} and \emph{evolutionary algorithms}. To 
reach a general understanding of these tools, a brief description of learning mechanisms
and numerical optimization is carried out.

\section{Computational Intelligence}
The first thing to address is the meaning and scope of both \emph{Computational 
Intelligence} (CI) and \emph{Machine Learning} (ML); in particular, when does one start
and the other end? With increasingly advancements in fundamental research in both areas,
there seems to be a blurry definition, and there is no concrete one until now. For this 
reason, in this work the definition of CI is an umbrella term for several other 
applications. However, these applications are related to each other for the same reason
that CI exists: to provide a computational solution to a problem using as inspiration
the paradigms of nature-inspired intelligence. For instance, following the handbook
by Kacprzyk and Pedrycz~\cite{kacprzykSpringerHandbookComputational2015},
the definition of CI is to be a collection of nature-inspired computational methods that
provide solutions to problems where \emph{hard computing} is inefficient or it not even
suited to provide a solution to a given problem. Here, there is an important distinction
that should be carried throughout the remaining of the work: that there are problems for
which traditional tools are insufficient for the traditional problems. In this work, this
is the philosophy used to provide solutions: that the traditional forms of solutions might
seem hard and unfitting to provide solutions to the problems presented, and therefore
new way of approaching these solutions should be used.

In most cases, CI methods do not provide exact and accurate results, but this is expected.
It does not mean that CI is providing the ultimate, best solution to a problem. Rather, it 
is providing an approximate solution that can be used later with more robust algorithms and
methods. CI is not meant to be used as the sole method to solve a problem, but instead to
help find a simple solution to a difficult problem.