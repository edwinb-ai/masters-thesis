\chapter{Numerical solution to the Ornstein-Zernike equation}
\label{AppendixB}

The Ornstein-Zernike (OZ) equation is usually solved using a particular closure for a given 
interaction potential. In this thesis, two closures were explored and discussed. However, 
the solution to the OZ equation remained the same. The solution is based on using the Fast 
Fourier Transform and the convolution theorem to solve algebraic equations instead of an 
integral equation. In this appendix, the complete numerical scheme used in this thesis will 
be described in detail, for the particular case of \(3\)-dimensional space.

The method used is a derivative of the so-called \emph{Piccard iterative} methods, with a 
variation due to Ng~\cite{ngHypernettedChainSolutions1974}, referred to in this work as the 
\emph{five-point method} of Ng. A slight variation was implemented in practice, which is 
just a straightforward extension to the Ng method.

\section{Fourier Transform of the Ornstein-Zernike equation}
The OZ for an isotropic fluid is, recalling from \autoref{eq:ornstein-zernike},
\begin{equation}
    h(r) = c(r) + \rho \int_{V} c(r') \, h(\lvert \vecr - \vecr' \rvert) \, d \vecr'
    \; ,
    \label{eq:oz-appendix}
\end{equation}
with \(\rho\) the particle number density, and \(h(r), c(r)\) are the total and direct correlation functions, respectively.

The \emph{Fourier transform} defined as,
\begin{equation}
    \hat{f}(k) = \int_{- \infty}^{\infty} f(x) \, e^{-2 \pi \, i k x} \, dx
    \; ,
    \label{eq:fourier-transform}
\end{equation}
and the \emph{inverse Fourier transform} is defined as,
\begin{equation}
    f(x) = \int_{- \infty}^{\infty} \hat{f}(k) \, e^{2 \pi \, i k x} \, dk
    \; .
    \label{eq:inv-fourier-transform}
\end{equation}

When using \autoref{eq:fourier-transform} in \autoref{eq:oz-appendix}, and using the 
\emph{convolution theorem}~\cite{kornerFourierAnalysis1989}, the result is the following 
algebraic equation,
\begin{equation}
    \hat{h}(k) = \hat{c}(k) + \rho \, \hat{c}(k) \, \hat{h}(k)
    \; .
    \label{eq:algebraic-oz}
\end{equation}
As is the case when dealing with Fourier transforms, solving \autoref{eq:algebraic-oz} is 
much easier than solving \autoref{eq:oz-appendix} directly. Not only because the equation 
is now an algebraic equation, but because there exist efficient numerical methods that can 
compute the Fourier transform in its discrete form.

\section{Piccard method}
The Piccard method is an iterative numerical method that can provide solutions to a 
\emph{linear system} based on previous iterations. The idea is to build a set of possible 
solutions that, when measured against each other, the error is minimized.

To see this more clearly, the OZ equation can be regarded as the linear system,
\begin{equation}
    A \, f = f
    \; ,
    \label{eq:linear-operator}
\end{equation}
with \(f \colon \mathbb{C} \mapsto \mathbb{C}\) is in general a complex-valued function, 
and \(A\) is a generic \emph{linear operator} defined on some functional space that acts on 
\(f\). In order to find the solutions to \autoref{eq:linear-operator}, the Piccard 
iterative method provides the following iterative algorithm,
\begin{equation}
    A \, f_{n+1} = f_{n}
    \; ,
    \label{eq:piccard}
\end{equation}
with \(n \in \mathbb{Z}\). When an initial function \(f_1\) is provided and plugged into
\autoref{eq:piccard}, several other functions are generated \(f_2, f_3, \dots\). If this 
sequence of functions converge to a particular limiting function \(g\), then that is 
defined as the solution to \autoref{eq:linear-operator}. However, in the case of fluids, 
and particularly liquids, the Piccard method often oscillates and diverges, thus no 
solutions is found. The method of Ng provides stability to the numerical method, and it 
shall be described next.

\section{The method of Ng}
Let \(g_n\) be defined as,
\begin{equation}
    g_n \coloneqq A \, f_n
    \: ,
    \label{eq:gn}
\end{equation}
and let \(d_n\) be defined as,
\begin{equation}
    d_n \coloneqq g_n - f_n = (A - 1) f_n
    \; .
    \label{eq:dn}
\end{equation}
In the numerical analysis jargon, function \(f_n\) is usually referred to as the 
\emph{input}, and function \(g_n\) is referred to as the \emph{output}. In both cases, 
\(n\) is the \(n\)-th iteration of the method. Further, \(d_n\) is usually employed as a 
\emph{metric} to measure the accuracy of the solution when obtaining its \(L^2\) norm in 
functional space, that is,
\begin{equation}
    {\left\lVert d_n \right\rVert}^{2} = \int {\left\lvert d_n (x) \right\rvert}^{2} \, dx
    \; .
    \label{eq:precision}
\end{equation}

Now, for the method of Ng, suppose that the following functions are known, for \(n \geq 6\),
\(f_{n-i}, g_{n-i}\) with \(i = 0,1,2,3,4,5 \, .\) The goal is to use all of these 
functions to generate a good estimation for the \emph{input} function to the iterative 
method, thus the following function is used,
\begin{equation}
    f=(1-c_{1}-c_{2}-c_{3}-c_{4}-c_{5}) f_n + \sum_{i=1}^{5} c_i f_{n-i}
    \; ,
    \label{eq:finit}
\end{equation}
with \(c_i\), \(i = 1,2,3,4,5\), arbitrary constants. Here, the goal is to find the values 
that will make \autoref{eq:finit} a good solution to \autoref{eq:linear-operator}.
Thus, pluggin \autoref{eq:finit} into \autoref{eq:linear-operator},
\begin{equation}
    A \, f = (1-c_{1}-c_{2}-c_{3}-c_{4}-c_{5}) g_n + \sum_{i=1}^{5} c_i g_{n-i}
    \; ,
    \label{eq:ginit}
\end{equation}
hence,
\begin{equation}
    \Delta \coloneqq \left\lVert A \, f - f \right\rVert =
    \left\lVert d_n - \sum_{i=1}^{5} c_i d_{0i} \right\rVert
    \; ,
    \label{eq:deltas}
\end{equation}
where
\begin{equation}
    d_{0i} = d_n - d_{n-i} \quad i=1,2,3,4,5
    \; .
    \label{eq:dzeros}
\end{equation}

Now, to find the best set of coefficients \(c_i\), the error \(\Delta^2\) must be minimized 
with respect to the coefficients \(c_i\), which gives,
\begin{equation}
    D_{ij} \cdot c_j = \left(d_n, d_{0i}\right)
    \; ,
    \label{eq:linear-system}
\end{equation}
where \(D_{ij} = \left(d_{0i}, d_{0j}\right)\) are the elements of the \(5 \times 5\) 
matrix, and the indices take the values \(i=j=1,2,3,4,5\). Here, \(\left(u, v\right)\) 
determines the \emph{inner product} defined to be,
\begin{equation}
    \left(u, v\right) = \int u(x) \, v(x) \, dx
    \: .
    \label{eq:inner-product}
\end{equation}

With this, the Ng method is now complete. The goal is to find the best values of \(c_i\) 
using \autoref{eq:linear-system} such that the \(n+1\)-th iteration of the \emph{input 
function}, \(f_{n+1}\),
\begin{equation}
    f_{n+1} = (1-c_{1}-c_{2}-c_{3}-c_{4}-c_{5}) g_n + \sum_{i=1}^{5} c_i g_{n-i}
    \; ,
    \label{eq:best-input}
\end{equation}
is the best approximation to solve \autoref{eq:linear-operator}.

\section{Numerical algorithm for solving the Ornstein-Zernike equation}
The numerical algorithm to solve the OZ equation will now be described, which uses the Ng 
method. For this, it is assumed that there is a \emph{closure relation} available that 
relates \(c(r)\) and \(\gamma(r) = h(r) - c(r)\) through a bridge function \(B(r)\). With 
this information, the numerical algorithm in \autoref{alg:cap} is used to solve the OZ equation.
\begin{algorithm}
    \caption{Piccard method for the OZ equation}\label{alg:cap}
    \begin{algorithmic}
        \State \(c(r) \gets \exp{\left[\beta u(r) + B(r) + \gamma(r)\right]} - \gamma(r) - 1\) \Comment{Using the bridge function} \\
        \State \(\mathcal{F} \left[c(r)\right] \gets \hat{c}(k)\) \\
        \State \(\hat{\gamma}(k) \gets \frac{\rho \, \hat{c}^{2}(k)}{1 - \rho \, \hat{c}(k)}\) \\
        \State \(\mathcal{F}^{-1} \left[\hat{\gamma}(k)\right] \gets \gamma^{\prime}(r)\) \\
        \State \(c^{\prime}(r) \gets \exp{\left[\beta u(r) + B(r) + \gamma^{\prime}(r)\right]} - \gamma(r) - 1\) \Comment{New \(c^{\prime}(r)\) using previous value of \(\gamma(r)\)}
    \end{algorithmic}
\end{algorithm}

In \autoref{alg:cap}, the symbol \(\mathcal{F} [f(x)]\) represents the Fourier transform of \(f(x)\), while \(\mathcal{F}^{-1} [\hat{f}(k)]\) represents the inverse Fourier transform of \(\hat{f}(k)\).
However, the solution to the OZ equation is quite sensible to the density of the system, 
which is why it makes it hard for the iterative method to converge properly, specially at 
high densities. A useful way of solving this issue is by constructing a grid of density 
values, and to solve the OZ equation at each intermediate value, until the target density 
is reached. This makes the method convergent, but at the cost of high computational 
resources, which stem from performing \autoref{alg:cap} as many times as there are density 
values in the grid. With this information, the complete algorithm will now be described 
next.

\begin{enumerate}
    \item \label{item:1} A grid of density values is built, \(\{\rho_i\} \, , i=1,2,\dots,M\); where \(\rho_{M}\) is the target density of the system, and \(\Delta n = n_{i+1} - n_{i}\).
    \item \label{item:2} Using the first density value \(n_1\) and \(\gamma_0 = 0\), \autoref{alg:cap} is used to obtain the functions \(\gamma_i (r) \, , i=1,2,3,4,5,6\). These functions are then mapped, \(\gamma_6 \mapsto g_n\), \(\gamma_5 \mapsto g_{n-1}\), \(\gamma_4 \mapsto g_{n-2}\), \(\gamma_3 \mapsto g_{n-3}\), \(\gamma_2 \mapsto g_{n-4}\), \(\gamma_1 \mapsto g_{n-5}\).
    \item \label{item:3} Then, using \autoref{eq:linear-system}, all the coefficients \(c_i\) are then computed, such that \autoref{eq:best-input} can then be obtained.
    \item \label{item:4} Using \autoref{eq:dn}, expression \(d_n = f_{n+1} - g_n\) is evaluated. If \(\lVert d_n \rVert > \num{1e-3}\), then the functions are updated as \(f_{n+1} \mapsto \gamma_6 \, , g_5 \mapsto \gamma_4 \, , g_4 \mapsto \gamma_3 \, , g_3 \mapsto \gamma_2 \, , g_2 \mapsto \gamma_1\). Then, the previous two steps are repeated until \(\lVert d_n \rVert \leq \num{1e-3}\).
    \item \label{item:5} Finally, once the condition \(\lVert d_n \rVert \leq \num{1e-3}\) is met, a new density value is taken \(n_2 = n_1 + \Delta n\), and the steps \ref{item:2} to \ref{item:4} are repeated. However, this time around, the initial \emph{input} is updated as \(\gamma_{0} \mapsto \gamma_6\), which resulted from the previous step. From here on, this process is repetead until convergence, which usually means until all the density values in the grid are used. However, the function update \(\gamma_{0} \mapsto \gamma_6\) is changed to be a linear extrapolation between the previous two results.
\end{enumerate}

\section{Extensions to the method}
In the actual method used in this work, the method of Ng is extended to use not only five 
input functions, but an arbitrary \(m\) number of functions. It turns out that 
this extension does not provide more accurate results, at least for the case of the hard 
sphere fluid. It does make it more stable for some high density values, numerically 
speaking. It would be interesting to see if the actual extension makes sense for other 
types of interaction potentials.

To solve the linear system in \autoref{eq:linear-system}, the common way of doing this is 
through the use of the LU decomposition~\cite{trefethenNumericalLinearAlgebra1997}. The 
main issue with the LU decomposition method is that is it slow, i.e., it needs a lot of 
floating point operations, and can also be numerically unstable when dealing with 
ill-conditioned matrices. Instead, in this work a faster method was employed which is based 
on the Krylov sub-space iterative family of methods. This method, called 
DIOM~\cite{saadPracticalUseKrylov1984a}, is an upgrade to the basic LU method, and can be 
used with ill-conditioned, symmetric matrices.