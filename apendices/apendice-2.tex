\chapter{Numerical solution to the Ornstein-Zernike equation}
\label{AppendixB}

The Ornstein-Zernike (OZ) equation is usually solved using a particular closure for a given 
interaction potential. In this thesis, two closures were explored and discussed. However, 
the solution to the OZ equation remained the same. The solution is based on using the Fast 
Fourier Transform and the convolution theorem to solve algebraic equations instead of an 
integral equation. In this appendix, the complete numerical scheme used in this thesis will 
be described in detail, for the particular case of \(3\)-dimensional space.

The method used is a derivative of the so-called \emph{Piccard iterative} methods, with a 
variation due to Ng~\cite{ngHypernettedChainSolutions1974}, referred to in this work as the 
\emph{five-point method} of Ng. A slight variation was implemented in practice, which is 
just a straightforward extension to the Ng method.

\section{Fourier Transform of the Ornstein-Zernike equation}
The OZ for an isotropic fluid is, recalling from \autoref{eq:ornstein-zernike},
\begin{equation}
    h(r) = c(r) + \rho \int_{V} c(r') \, h(\lvert \vecr - \vecr' \rvert) \, d \vecr'
    \; ,
    \label{eq:oz-appendix}
\end{equation}
with \(\rho\) the particle number density, and \(h(r), c(r)\) are the total and direct correlation functions, respectively.

The \emph{Fourier transform} defined as,
\begin{equation}
    \hat{f}(k) = \int_{- \infty}^{\infty} f(x) \, e^{-2 \pi \, i k x} \, dx
    \; ,
    \label{eq:fourier-transform}
\end{equation}
and the \emph{inverse Fourier transform} is defined as,
\begin{equation}
    f(x) = \int_{- \infty}^{\infty} \hat{f}(k) \, e^{2 \pi \, i k x} \, dk
    \; .
    \label{eq:inv-fourier-transform}
\end{equation}

When using \autoref{eq:fourier-transform} in \autoref{eq:oz-appendix}, and using the 
\emph{convolution theorem}~\cite{kornerFourierAnalysis1989}, the result is the following 
algebraic equation,
\begin{equation}
    \hat{h}(k) = \hat{c}(k) + \rho \, \hat{c}(k) \, \hat{h}(k)
    \; .
    \label{eq:algebraic-oz}
\end{equation}
As is the case when dealing with Fourier transforms, solving \autoref{eq:algebraic-oz} is 
much easier than solving \autoref{eq:oz-appendix} directly. Not only because the equation 
is now an algebraic equation, but because there exist efficient numerical methods that can 
compute the Fourier transform in its discrete form.