\chapter{Numerical solution to the Ornstein-Zernike equation}
\label{AppendixB}

The Ornstein-Zernike (OZ) equation is usually solved using a particular closure for a given 
interaction potential. In this thesis, two closures were explored and discussed. However, 
the solution to the OZ equation remained the same. The solution is based on using the Fast 
Fourier Transform and the convolution theorem to solve algebraic equations instead of an 
integral equation. In this appendix, the complete numerical scheme used in this thesis will 
be described in detail, for the particular case of \(3\)-dimensional space.

The method used is a derivative of the so-called \emph{Piccard iterative} methods, with a 
variation due to Ng~\cite{ngHypernettedChainSolutions1974}, referred to in this work as the 
\emph{five-point method} of Ng. A slight variation was implemented in practice, which is 
just a straightforward extension to the Ng method.

\section{Fourier Transform of the Ornstein-Zernike equation}
The OZ for an isotropic fluid is, recalling from \autoref{eq:ornstein-zernike},
\begin{equation}
    h(r) = c(r) + \rho \int_{V} c(r') \, h(\lvert \vecr - \vecr' \rvert) \, d \vecr'
    \; ,
    \label{eq:oz-appendix}
\end{equation}
with \(\rho\) the particle number density, and \(h(r), c(r)\) are the total and direct correlation functions, respectively.

The \emph{Fourier transform} defined as,
\begin{equation}
    \hat{f}(k) = \int_{- \infty}^{\infty} f(x) \, e^{-2 \pi \, i k x} \, dx
    \; ,
    \label{eq:fourier-transform}
\end{equation}
and the \emph{inverse Fourier transform} is defined as,
\begin{equation}
    f(x) = \int_{- \infty}^{\infty} \hat{f}(k) \, e^{2 \pi \, i k x} \, dk
    \; .
    \label{eq:inv-fourier-transform}
\end{equation}

When using \autoref{eq:fourier-transform} in \autoref{eq:oz-appendix}, and using the 
\emph{convolution theorem}~\cite{kornerFourierAnalysis1989}, the result is the following 
algebraic equation,
\begin{equation}
    \hat{h}(k) = \hat{c}(k) + \rho \, \hat{c}(k) \, \hat{h}(k)
    \; .
    \label{eq:algebraic-oz}
\end{equation}
As is the case when dealing with Fourier transforms, solving \autoref{eq:algebraic-oz} is 
much easier than solving \autoref{eq:oz-appendix} directly. Not only because the equation 
is now an algebraic equation, but because there exist efficient numerical methods that can 
compute the Fourier transform in its discrete form.

\section{Piccard method}
The Piccard method is an iterative numerical method that can provide solutions to a 
\emph{linear system} based on previous iterations. The idea is to build a set of possible 
solutions that, when measured against each other, the error is minimized.

To see this more clearly, the OZ equation can be regarded as the linear system,
\begin{equation}
    A \, f = f
    \; ,
    \label{eq:linear-operator}
\end{equation}
with \(f \colon \mathbb{C} \mapsto \mathbb{C}\) is in general a complex-valued function, 
and \(A\) is a generic \emph{linear operator} defined on some functional space that acts on 
\(f\). In order to find the solutions to \autoref{eq:linear-operator}, the Piccard 
iterative method provides the following iterative algorithm,
\begin{equation}
    A \, f_{n+1} = f_{n}
    \; ,
    \label{eq:piccard}
\end{equation}
with \(n \in \mathbb{Z}\). When an initial function \(f_1\) is provided and plugged into
\autoref{eq:piccard}, several other functions are generated \(f_2, f_3, \dots\). If this 
sequence of functions converge to a particular limiting function \(g\), then that is 
defined as the solution to \autoref{eq:linear-operator}. However, in the case of fluids, 
and particularly liquids, the Piccard method often oscillates and diverges, thus no 
solutions is found. The method of Ng provides stability to the numerical method, and it 
shall be described next.

\section{The method of Ng}
Let \(g_n\) be defined as,
\begin{equation}
    g_n \coloneqq A \, f_n
    \: ,
    \label{eq:gn}
\end{equation}
and let \(d_n\) be defined as,
\begin{equation}
    d_n \coloneqq g_n - f_n = (A - 1) f_n
    \; .
    \label{eq:dn}
\end{equation}
In the numerical analysis jargon, function \(f_n\) is usually referred to as the 
\emph{input}, and function \(g_n\) is referred to as the \emph{output}. In both cases, 
\(n\) is the \(n\)-th iteration of the method. Further, \(d_n\) is usually employed as a 
\emph{metric} to measure the accuracy of the solution when obtaining its \(L^2\) norm in 
functional space, that is,
\begin{equation}
    {\left\lVert d_n \right\rVert}^{2} = \int {\left\lvert d_n (x) \right\rvert}^{2} \, dx
    \; .
    \label{eq:precision}
\end{equation}

Now, for the method of Ng, suppose that the following functions are known, for \(n \geq 6\),
\(f_{n-i}, g_{n-i}\) with \(i = 0,1,2,3,4,5 \, .\) The goal is to use all of these 
functions to generate a good estimation for the \emph{input} function to the iterative 
method, thus the following function is used,
\begin{equation}
    f=(1-c_{1}-c_{2}-c_{3}-c_{4}-c_{5}) f_n + \sum_{i=1}^{5} c_i f_{n-i}
    \; ,
    \label{eq:finit}
\end{equation}
with \(c_i\), \(i = 1,2,3,4,5\), arbitrary constants. Here, the goal is to find the values 
that will make \autoref{eq:finit} a good solution to \autoref{eq:linear-operator}.
Thus, pluggin \autoref{eq:finit} into \autoref{eq:linear-operator},
\begin{equation}
    A \, f = (1-c_{1}-c_{2}-c_{3}-c_{4}-c_{5}) g_n + \sum_{i=1}^{5} c_i g_{n-i}
    \; ,
    \label{eq:ginit}
\end{equation}
hence,
\begin{equation}
    \Delta \coloneqq \left\lVert A \, f - f \right\rVert =
    \left\lVert d_n - \sum_{i=1}^{5} c_i d_{0i} \right\rVert
    \; ,
    \label{eq:deltas}
\end{equation}
where
\begin{equation}
    d_{0i} = d_n - d_{n-1} \quad i=1,2,3,4,5
    \; .
    \label{eq:dzeros}
\end{equation}

Now, to find the best set of coefficients \(c_i\), the error \(\Delta^2\) must be minimized 
with respect to the coefficients \(c_i\), which gives,
\begin{equation}
    D_{ij} \cdot c_j = \left(d_n, d_{0i}\right)
    \; ,
    \label{eq:linear-system}
\end{equation}
where \(D_{ij} = \left(d_{0i}, d_{0j}\right)\) are the elements of the \(5 \times 5\) 
matrix, and the indices take the values \(i=j=1,2,3,4,5\). Here, \(\left(u, v\right)\) 
determines the \emph{inner product} defined to be,
\begin{equation}
    \left(u, v\right) = \int u(x) \, v(x) \, dx
    \: .
    \label{eq:inner-product}
\end{equation}

With this, the Ng method is now complete. The goal is to find the best values of \(c_i\) 
using \autoref{eq:linear-system} such that the \(n+1\)-th iteration of the \emph{input 
function}, \(f_{n+1}\),
\begin{equation}
    f_{n+1} = (1-c_{1}-c_{2}-c_{3}-c_{4}-c_{5}) g_n + \sum_{i=1}^{5} c_i g_{n-i}
    \; ,
    \label{eq:best-input}
\end{equation}
is the best approximation to solve \autoref{eq:linear-operator}.